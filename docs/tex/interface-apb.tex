\section{APB4 (Peripheral)
Interface}\label{apb4-peripheral-interface}

The APB4Interface is a regular APB4 Master Interface. All signals
defined in the protocol are supported as described below. See the
\emph{AMBA APB Protocol v2.0 Specifications} for a complete description
of the signals.

\begin{longtable}[]{@{}lccl@{}}
\toprule
Port & Size & Direction & Description\tabularnewline
\midrule
\endhead
\texttt{PRESETn} & 1                      & Input  & Asynchronous active low reset\tabularnewline
\texttt{PCLK}    & 1                      & Input  & Clock Input\tabularnewline
\texttt{PSEL}    & 1                      & Output & Peripheral Select\tabularnewline
\texttt{PENABLE} & 1                      & Output & Peripheral Enable Control\tabularnewline
\texttt{PPROT}   & 3                      & Output & Transfer Protection Level\tabularnewline
\texttt{PWRITE}  & 1                      & Output & Write Select\tabularnewline
\texttt{PSTRB}   & \texttt{PDATA\_SIZE/8} & Output & Byte Lane Indicator\tabularnewline
\texttt{PADDR}   & \texttt{PADDR\_SIZE}   & Output & Address Bus\tabularnewline
\texttt{PWDATA}  & \texttt{PDATA\_SIZE}   & Output & Write Data Bus\tabularnewline
\texttt{PRDATA}  & \texttt{PDATA\_SIZE}   & Input  & Read Data Bus\tabularnewline
\texttt{PREADY}  & 1                      & Input  & Transfer Ready Input\tabularnewline
\texttt{PSLVERR} & 1                      & Input  & Transfer Error Indicator\tabularnewline
\bottomrule
\caption{APB4 Peripheral Interface Ports}
\end{longtable}

\subsection{PRESETn}\label{presetn}

When the active low asynchronous \texttt{PRESETn} input is asserted (`0'), the
APB4 interface is put into its initial reset state.

\subsection{PCLK}\label{pclk}

\texttt{PCLK} is the APB4 interface system clock. All internal logic for the APB4
interface operates at the rising edge of this system clock and APB4 bus
timings are related to the rising edge of \texttt{PCLK}.

The frequency of \texttt{PCLK} must be less than or equal to that of the AHB-Lite
Interface clock \texttt{HCLK}:

\begin{longtable}[]{@{}|rp{12cm}@{}}
	\textbf{Conditions}: & \\
	\endhead
	 & Freq( \texttt{PCLK} ) $\leqslant$ Freq( \texttt{HCLK} )\\
\end{longtable}	

\subsection{PSEL}\label{psel}

The APB4 Bridge generates \texttt{PSEL}, signaling to an attached peripheral that
it is selected and a data transfer is pending.

\begin{longtable}[]{@{}|p{2cm}p{12cm}@{}}
\textbf{Note:} & \\
\endhead
& To support multiple APB4 peripherals, individual \texttt{PSEL}
signals must be generated per peripheral - Roa Logic provides an
additional \emph{APB4 Multiplexer} IP to support this
requirement\tabularnewline

\end{longtable}

\subsection{PENABLE}\label{penable}

The APB4 Bridge asserts \texttt{PENABLE} during the second and
subsequent cycles of an APB4 data transfer.

\subsection{PPROT}\label{pprot}

\texttt{PPROT[2:0]} indicates the protection type of the data transfer, with
3 levels of protection supported as follows:

\begin{longtable}[]{@{}lll@{}}
\toprule
Bit\# & Value & Description\tabularnewline
\midrule
\endhead
2 & 1 & Instruction Access\tabularnewline
  & 0 & Data Access\tabularnewline
1 & 1 & Non-Secure Access\tabularnewline
  & 0 & Secure Access\tabularnewline
0 & 1 & Privileged Access\tabularnewline
  & 0 & Normal Access\tabularnewline
\bottomrule
\caption{APB4 Protection Types}
\end{longtable}

\subsection{PWRITE}\label{pwrite}

\texttt{PWRITE} indicates a data write access when asserted high (`1') and a read
data access when de-asserted (`0')

\subsection{PSTRB}\label{pstrb}

There is one \texttt{PSTRB} signal per byte lane of the APB4 write data bus
(\texttt{PWDATA}). These signals indicate which byte lane to update during a
write transfer such that \texttt{PSTRB[n]} corresponds to
\texttt{PWDATA[(8n+7):8n]}.

\subsection{PADDR}\label{paddr}

\texttt{PADDR} is the APB4 address bus. The bus width is defined by the
\texttt{PADDR\_SIZE} parameter and is driven by the APB4 Bridge core.

\subsection{PWDATA}\label{pwdata}

\texttt{PWDATA} is the APB4 write data bus and is driven by the APB4 Bridge core
during write cycles, indicated when \texttt{PWRITE} is asserted (`1'). The bus
width must be byte-aligned and is defined by the \texttt{PDATA\_SIZE} parameter.

\subsection{PRDATA}\label{prdata}

\texttt{PRDATA} is the APB4 read data bus. An attached peripheral drives this bus
during read cycles, indicated when \texttt{PWRITE} is de-asserted (`0'). The bus
width must be byte-aligned and is defined by the \texttt{PDATA\_SIZE} parameter.

\subsection{PREADY}\label{pready}

\texttt{PREADY} is driven by the attached peripheral. It is used to extend an
APB4 transfer.

\subsection{PSLVERR}\label{pslverr}

\texttt{PSLVERR} indicates a failed data transfer when asserted (`1'). As APB4
peripherals are not required to support this signal it must be tied LOW
(`0') when unused.